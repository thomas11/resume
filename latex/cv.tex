\selectlanguage{english}

\renewcommand{\thefootnote}{\fnsymbol{footnote}}

\begin{center}
{\large
\textbf{Thomas Kappler}
}
\end{center}

% \vspace{0.1cm}


\begin{center}
\begin{tabular}[b]{ll}
  \emph{Email} & tkappler@gmail.com\\
%  \emph{Address} & Bankstrasse 17, CH-8610 Uster\\
  \emph{Phone} & +41 78 844 60 77 (mobile)\\
%  \emph{Date of birth} & November 11th, 1980\\
  \emph{Nationality} & German, US green card expected Nov 2015\\
  \emph{Web} & \href{http://www.thomaskappler.net/}{www.thomaskappler.net}
\end{tabular}
\end{center}

\vspace{0.2cm}

\begin{cv}{}
  \begin{center}
    \begin{varwidth}[t]{0.8\linewidth}
      \raggedright Passionate Software Engineer with eight years of
      experience across a wide range of technologies, including
      excellent skills in Java, Go, SQL and NoSQL databases,
      Elasticsearch, Linux, and Azure.
    \end{varwidth}
  \end{center}

% have
    % published open source code and presented at conferences, and have
    % startup experience with a track record of delivering software in a
    % dynamic environment.

    % \begin{itemize}
    % \item Software Engineer with \textbf{six years experience in
    %     software development} in Java, Go, JavaScript, Python, Perl,
    %   SQL and NoSQL databases, web development, unit testing.
    % % \item Developed \textbf{web applications} with Dojo, Jetty, Apache Wicket,
    %   % Ajax, JSON, REST, HTML and CSS.
    % \item Co-authored peer-reviewed publications and software in
    %   \textbf{Text Mining}.
    % \item Startup experience with a track record of delivering
    %   software in a dynamic environment.
    % \item Published \textbf{Open Source} code and presented at
    %   \textbf{conferences}.
    % %\item Worked as assistant nurse in a hospital
    %   % \item Participating in the international scientific bioinformatics
    %   %   community
    % % \item Additional expertise in Bioinformatics and Linux
    % %  administration
    % \end{itemize}    
    % \end{minipage}

  \vspace{0.15cm}

  \begin{cvlist}{Work Experience}
  \item[2013-08--present]
    \textbf{\href{http://www.microsoft.com/}{Microsoft}}, Zürich:
    Software Development Engineer (II).
    \begin{itemize}
    \item Back-end developer and team lead on
      \href{http://www.microsoft.com/en-us/dynamics/crm-social.aspx}{Microsoft
        Social Engagement}, using Java, Elasticsearch, Lucene, and
      Azure including SQL, Service Bus, Blob Store, VMs.
    \item Co-author of two patents for determining significant phrases
      from text (in submission).\\Implemented on top of Elasticsearch.
    \item Attended MS Spring 2015 Machine Learning \& Data Science conference
    \end{itemize}
  \item[2011-04--2013-07] \textbf{\href{http://nhumi.com/}{Nhumi
        Technologies}} and
    \textbf{\href{http://www.zhaw.ch/en/zurich-university-of-applied-sciences.html}{Zürcher
        Hochschule für Angewandte Wis\-sen\-schaf\-ten (zhaw)}},
    Zürich: Software Engineer.
    \begin{itemize}
    \item Full-stack developer on agile startup team of four.
    \item Developed and delivered medical applications such as
      \href{http://healthcorpus.com}{healthcorpus} in Java,
      JavaScript, Go, Python, Hadoop and Pig, SQL (sqlite, Postgres),
      MongoDB, LevelDB.
    \item Responsible for data import, processing and quality of
      \href{http://drugsafety.nhumi.com/drugsafety/}{Nhumi Drug
        Safety}.
    \item Technical lead for Nhumi on the KTI research grant ``NoSQL
      Data Warehouse'' with zhaw.
    %\item Co-designed and implemented complex, concurrent solutions
    %  such as a fast in-memory database.
    \end{itemize}
  \item[2008-05--2011-04] \textbf{\href{http://www.isb-sib.ch/}{Swiss
        Institute of Bioinformatics}}, Geneva: Developer.
    \begin{itemize}
    \item One of four developers on
      \href{http://www.uniprot.org/}{UniProt.org}, bioinformatics resource
      with 30,000 visitors/day.
    \item Developed web applications from frontend to backend, in Java
      and JavaScript, using relational databases, Lucene, Berkeley DB,
      Wicket, Struts, jQuery, JSON.
    %\item Introduced unit testing for Perl. Tested Java code with
    %  JUnit, Selenium, JMock, Hudson.
    % \item Optimized performance and memory in both Java and Perl,
    %   using JProf, Eclipse Memory Analyzer, Devel::NYTProf, among others.
    % \item Developed RDF vocabularies and processing for one of the
    %  largest free RDF data sets.
    % \item Evolved domain-specific vocabularies together with the
    %   scientific community.
    % \item Presented at ``Semantic Web: Applications and Tools for the
    %   Life Sciences'' 2009, Biohackathon 2010, German Perl Workshop
    %   2010, FrOSCamp Zurich 2010. Gave internal presentations on best
    %   practices, Git, and Perl.
    \item Presented at ``Semantic Web: Applications and Tools for the
      Life Sciences'' 2009, Biohackathon 2010 Tokyo, German Perl
      Workshop 2010. Gave internal training on Git and Perl best
      practices.
    \end{itemize}
  \item[2007-10--2008-04] \textbf{University of Heidelberg}: Freelance
    developer.
    \begin{itemize}
    \item Developed an application (25 KLOC) for thesaurus editing,
      in Java using Wicket and Hibernate.
    \item Sole full-time developer from requirements analysis to
      delivering the first production version.
    \item Led two part-time programmers.
    \end{itemize}
  \item[2005-07--2006-07] Karlsruhe Institute of Technology: Developer and
    research assistant.
    \begin{itemize}
    \item Wrote parsers, database and ontology tools for NLP research
      in Java and Python.
    % \item Implemented components for the GATE text analysis framework
      % in Java.
    % \item Wrote parsers and database tools in Python (11
      % KLOC). Designed schemas and OWL ontologies.
    \end{itemize}
  \item[2004-05--2004-10] ZKM, Karlsruhe: PHP and SQL developer,
    Linux/Apache admin.
  \item[2000-08--2001-06] Rommel-Klinik hospital, Bad Wildbad:
    Civilian service as assistant nurse, working with severely
    disabled patients.
  \end{cvlist}

  \pagebreak

  \begin{cvlist}{Open Source, Talks}
  \item[] \emph{Mostly published on
      \href{http://github.com/thomas11}{github.com/thomas11}.}
  % \item[Online courses:] ``Machine Learning'', Coursera,
  %   with Andrew Ng, Stanford. Score 100\% (2013).
  %   \\``Introduction to Artificial Intelligence'', Stanford. Score 88\% (2012).
  \item[Go:] Core contributor (bug fixes),
    \emph{\href{https://github.com/thomas11/atomgenerator}{atomgenerator}}
    Atom feed package,
    \emph{\href{https://github.com/thomas11/csp}{CSP}} concurrency
    package, ``Intro to Go'' presentation at Google Developer Group
    DevFest Zurich 2012.
  \item[Meetups:] Co-organizer of
    \href{http://www.meetup.com/zhgeeks/}{zhgeeks}, a monthly
    technical meetup in Zurich. Recent subjects included Go, Riak
    architecture, Neo4J and AWS auto-scaling.
  % \item[Apache Wicket:] Contributed a visual, database-backed tree
    % component to databinder, a bridge from Wicket to
    % Hibernate. Patches in Wicket Web Beans.
  \item[Emacs:] Several extensions.
  \item[Perl:] Presented at German Perl Workshop 2010 and FrOSCamp
    Zurich 2010 about RDF in Perl.
  % \item[JavaScript:] jquery-tablesorter-filter extension, unit testing
    % tools\footnotemark[\value{footnote}].
  \end{cvlist}

  \begin{cvlist}{Education}
  \item[2007-09] \textbf{Diplom-Informatiker},
    corresponding to \textbf{Master of Computer Science}, grade
    1.4\footnote{German grades range from 1 (best) to 4 (worst).}
    (very good). \textbf{Karlsruhe Institute of Technology}, Germany.
  \item[2007-02--2007-08] Master's thesis at \textbf{IIT Delhi},
    India, with Prof. Pankaj Jalote and Prof. Ralf Reussner, with a
    DAAD (German Academic Exchange Service) scholarship.
    % Topic: \emph{Code-Analysis
    %   Using Eclipse to Support Performance Prediction for Java
    %   Components}.
    Implemented an Eclipse plug-in to reverse engineer
    Java code.  Grade~1.3 (very good)\footnotemark[\value{footnote}].
  \item[2001-10--2007-09] Karlsruhe Institute of Technology, Germany: Computer
    Science.  \\Focus on Information Retrieval (grade
    1.3\footnotemark[\value{footnote}]) and Software Engineering
    (1.0).  \\Other subjects were education \& pedagogics (1.7), and
    genetics (2.0).
  \item[2006-08] 18th European Summer School in Logic, Language and
    Information, Malaga, Spain.
%  \item[2000-06] Gymnasium Neuenbuerg, Germany: Abitur as
%    secondary school di\-plo\-ma.  Grade~1.4.
  \end{cvlist}

%\pagebreak
  \begin{cvlist}{Publications (selection)}
    \item[2011] N. Naderi, T. Kappler, Ch. J. O. Baker, and R.
      Witte. \emph{OrganismTagger: Detection, Normalization, and
        Grounding of Organism Entities in Biomedical Documents}.
      Bioinformatics, vol. 27.
  \item[2009] J. Bolleman and T. Kappler. \emph{Weekend Triple
      Billionaire}. Semantic Web: Applications and Tools for the Life
    Sciences 2009, Amsterdam.
  % \item[2008-06] R. Witte, T. Gitzinger, T. Kappler, and
    % R. Krestel. \emph{Semantic Wiki Approach to Cultural Heritage Data
      % Management}. LaTeCH workshop at LREC 2008, Morocco.
  \item[2007] T. Kappler, H. Koziolek, K. Krogmann, and
    R. Reussner.  \emph{Towards Automatic Construction of Reusable
      Prediction Models for Component-Based Performance Engineering}.
    Proc. Software Engineering 2008, ser. LNI, vol. 121.
  % \item[2007-09] R. Witte, T. Kappler, and
  %   Ch. J. O. Baker. \emph{Enhanced Semantic Access to the Protein
  %     Engineering Literature using Ontologies Populated by Text
  %     Mining}.  International Journal of Bioinformatics Research and
  %   Applications (IJBRA), 3(3):389–413, 2007.
  \item[2006] R. Witte, T. Kappler, and Ch. J. O. Baker.
    \emph{Ontology Design for Biomedical Text Mining} in
    \emph{Semantic Web: Revolutionizing Knowledge Discovery in the
      Life Sciences}, Springer.
  % \item[2005-11] R. Witte, P. Gerlach, M. Joachim, T. Kappler,
  %   R. Krestel, and P. Perera.  \emph{Engineering a Semantic Desktop
  %     for Building Historians and Architects}. Proceedings of the
  %   Semantic Desktop Workshop at the ISWC, volume 175 of CEUR Workshop
  %   Proc., p. 138–152, 2005.
  \end{cvlist}

  \begin{cvlist}{Other Skills}
    % \item
    %   \begin{description}
  \item[Coursera online course:] \vspace{-1.2mm}``Machine Learning''
    with Andrew Ng, Stanford, with score of 100\% (2013).
  \item[Stanford online course:] \vspace{-1.2mm}``Introduction to Artificial
    Intelligence'' with score of 88\% (2012).
  \item[Programming:] \vspace{-1.2mm}Non-professional use of Clojure, R,
    Erlang, C.
  \item[Version control:] \vspace{-1.2mm}Used Git, Mercurial, Subversion, and CVS
    professionally.
  \item[Linux:] \vspace{-1.2mm}LPI Certified Linux Administrator,
    junior level (2006).
      % \item[Job training] Vocational training certification (``AdA'') by
      %   the German CCI.
  \item[Languages:] \vspace{-1.2mm}Native German, excellent English, good
    French.
      % \end{description}
  \end{cvlist}

  % \begin{cvlist}{Other Activities}
  % \item[Tutor] Photography tutor at a student hostel.  This elected
  %   task included giving courses on darkroom work and caring for the
  %   darkroom.
  % \item[Orchestra] Musician and member of the organization committee
  %   of an amateur orchestra.  This elected task included help in
  %   organizing concerts and representing the younger musicians.
  %   Created and administrated a website for the orchestra.
  % \item[Hobbies] Hiking, reading, programming.
  % \end{cvlist}

  \begin{cvlist}{References}
  \item[] \hspace{-3.5mm}Ralf Gautschi: current manager. rgaut@microsoft.com.
  \item[] \hspace{-3.5mm}More references gladly provided on request.

  % \item[] \hspace{-4.5mm}Prof. Ralf Reussner, University of Karlsruhe (thesis,
  %   India exchange);
  %   \texttt{reussner@ipd.uka.de}.
  \end{cvlist}

%  \cvplace{References gladly provided on request.\\\\Uster, Switzerland}

\end{cv}

% \pagebreak
% \pagestyle{empty}
% \begin{figure}
%   \vspace*{-3.3cm}
%   \hspace{-1in}\hspace{-\oddsidemargin}
%   \includegraphics{../Scans/swiss-prot-reference}
% \end{figure}
